% THIS IS SIGPROC-SP.TEX - VERSION 3.1
% WORKS WITH V3.2SP OF ACM_PROC_ARTICLE-SP.CLS
% APRIL 2009
%
% It is an example file showing how to use the 'acm_proc_article-sp.cls' V3.2SP
% LaTeX2e document class file for Conference Proceedings submissions.
% ----------------------------------------------------------------------------------------------------------------
% This .tex file (and associated .cls V3.2SP) *DOES NOT* produce:
%       1) The Permission Statement
%       2) The Conference (location) Info information
%       3) The Copyright Line with ACM data
%       4) Page numbering
% ---------------------------------------------------------------------------------------------------------------
% It is an example which *does* use the .bib file (from which the .bbl file
% is produced).
% REMEMBER HOWEVER: After having produced the .bbl file,
% and prior to final submission,
% you need to 'insert'  your .bbl file into your source .tex file so as to provide
% ONE 'self-contained' source file.
%
% Questions regarding SIGS should be sent to
% Adrienne Griscti ---> griscti@acm.org
%
% Questions/suggestions regarding the guidelines, .tex and .cls files, etc. to
% Gerald Murray ---> murray@hq.acm.org
%
% For tracking purposes - this is V3.1SP - APRIL 2009

\documentclass{edm_template}

\begin{document}

\title{YouEDU: Addressing Confusion in MOOC Discussion Forums by Recommending Instructional Video Clips}
%\subtitle{[Extended Abstract]
%\titlenote{A full version of this paper is available as
%\textit{Author's Guide to Preparing ACM SIG Proceedings Using
%\LaTeX$2_\epsilon$\ and BibTeX} at
%\texttt{www.acm.org/eaddress.htm}}}
%
% You need the command \numberofauthors to handle the 'placement
% and alignment' of the authors beneath the title.
%
% For aesthetic reasons, we recommend 'three authors at a time'
% i.e. three 'name/affiliation blocks' be placed beneath the title.
%
% NOTE: You are NOT restricted in how many 'rows' of
% "name/affiliations" may appear. We just ask that you restrict
% the number of 'columns' to three.
%
% Because of the available 'opening page real-estate'
% we ask you to refrain from putting more than six authors
% (two rows with three columns) beneath the article title.
% More than six makes the first-page appear very cluttered indeed.
%
% Use the \alignauthor commands to handle the names
% and affiliations for an 'aesthetic maximum' of six authors.
% Add names, affiliations, addresses for
% the seventh etc. author(s) as the argument for the
% \additionalauthors command.
% These 'additional authors' will be output/set for you
% without further effort on your part as the last section in
% the body of your article BEFORE References or any Appendices.

\numberofauthors{3} %  in this sample file, there are a *total*
% of EIGHT authors. SIX appear on the 'first-page' (for formatting
% reasons) and the remaining two appear in the \additionalauthors section.
%
\author{
% You can go ahead and credit any number of authors here,
% e.g. one 'row of three' or two rows (consisting of one row of three
% and a second row of one, two or three).
%
% The command \alignauthor (no curly braces needed) should
% precede each author name, affiliation/snail-mail address and
% e-mail address. Additionally, tag each line of
% affiliation/address with \affaddr, and tag the
% e-mail address with \email.
%
% 1st. author
\alignauthor Akshay Agrawal\\
       \affaddr{Stanford University}\\
       \email{akshayka@cs.stanford.edu}
% 2nd. author
\alignauthor Jagadish Venkatraman\\
       \affaddr{Stanford University}\\
       \email{vjagadish@cs.stanford.edu}
% 3rd. author
\alignauthor Andreas Paepcke\\
       \affaddr{Stanford University}\\
       \email{paepcke@cs.stanford.edu}
% \and   use '\and' if you need 'another row' of author names
}
\date{9 February 2015}
% Just remember to make sure that the TOTAL number of authors
% is the number that will appear on the first page PLUS the
% number that will appear in the \additionalauthors section.

\maketitle
\begin{abstract}
In Massive Open Online Courses (MOOCs), struggling learners often seek help by
posting questions in discussion forums. Unfortunately, given the large volume of
discussion in MOOCs, instructors may overlook these learners' posts,
detrimentally impacting the learning process and exacerbating attrition. In this
paper, we present YouEDU, an instructional aid that automatically detects and
addresses confusion in forum posts. Leveraging our publicly-available Stanford
MOOCPosts corpus, we train a heterogeneous set of classifiers to classify forum
posts across multiple dimensions. In particular, classifiers that target
sentiment, urgency, and other descriptive variables inform a single classifier
that detects confusion. We then employ information retrieval techniques to map
confused posts to minute-resolution clips from course videos; the ranking over
these clips accounts for both video-clickstream data and textual similarity
between posts and closed captions. We measure the performance of our
classification model in multiple educational contexts, exploring the nature of
confusion within each; we also evaluate the relevancy of materials returned by
our ranking algorithm.
\end{abstract}

%% A category with the (minimum) three required fields
%\category{H.4}{Information Systems Applications}{Miscellaneous}
%%A category including the fourth, optional field follows...
%\category{D.2.8}{Software Engineering}{Metrics}[complexity measures, performance measures]
%
%\terms{Theory}

\keywords{ACM proceedings, \LaTeX, text tagging} % NOT required for Proceedings

\section{Introduction}
The remainder of this paper examines related work (section two), presents the Stanford MOOCPosts corpus (section three), sketches the architecture of YouEDU(section four), details its constituent classification and recommendation phases, evaluating both and interpreting results (sections five and six), and proposes future work (section seven).

\section{Related Work}
The remainder of this document is concerned with showing, in
the context of an ``actual'' document, the \LaTeX\ commands
specifically available for denoting the structure of a
proceedings paper, rather than with giving rigorous descriptions
or explanations of such commands.

\section{The Stanford MOOCPosts Corpus}
In order to enable this and similar research, we compiled and curated the Stanford MOOCPosts Dataset ...

\subsection{Methodology: Compiling the Dataset}
Nine judges from oDesk were hired to ...

\subsection{Insights and Discussion}
We report insights gleaned into the nature of affect, etc. across these courses.

% TODO: Is this really the right place to present this? Or should this
% be presented in the Classification Combination Section?
\subsubsection{Relationship between Variables}
In this section, we report the pairwise correlations between variables to 1) shed some light into the nature of each and also 2) to motivate a YouEDU design choice.

\section{YouEDU: Intelligent Interventions}
As the name might imply, YouEDU is an intervention system with granularity at the level of the individual ...

\section{Phase I: Detecting Confusion}
In this section, we present the model used to classify confusion. At a high-level, we use a coordinated combination of classifiers ...

\subsection{Classifier Design}
\subsubsection{Feature Space}
Here's an intuition of what we figured might be helpful across all variables ...

* Bag of Words \\
(* Specific Bigrams / words / trigrams? Time permitting) \\
* Pre-processing Steps \\
* Features Extracted / Generated \\

\subsubsection{A Combination of Hypotheses}
* Classifier Combination -- Overview \\
* Our particular implementation \\
* Subclassifiers ~ Variable-Specific Features \\
* Training, use gold values, testing, use predicted values \\
* Combination Step ~ Logistic Regression Layer \\

\subsection{Evaluation}

* educational contexts \\
* metrics used \\ 
* results \\
* implications \\

\section{Phase II: Mapping Confused Posts to Video Clips}
\subsection{The Recommendation Algorithm}
\subsubsection{Retrieval}
\subsubsection{Ranking}

\subsection{Evaluation}
Two experts were hired ...

\section{Future Work}
Future work might focus on strengthening the link between the classifiers and the reccomendation system; in particular, it would behoove us to devise a way to filter our set of confused posts to a subset for which recommendation makes sense. Additionally, we might want to make our classifiers better and index back into the previous course to retrieve answers for courses. Deploying this system live is another thing that we might do. 

\section{Conclusion}
YouEDU takes an initial step towards building automated confusion intervention ... 

%\end{document}  % This is where a 'short' article might terminate

%ACKNOWLEDGMENTS are optional
\section{Acknowledgments}
This section is optional\cite{wen2014sentiment}; it is a location for you
to acknowledge grants, funding, editing assistance and
what have you.  In the present case, for example, the
authors would like to thank Gerald Murray of ACM for
his help in codifying this \textit{Author's Guide}
and the \textbf{.cls} and \textbf{.tex} files that it describes.

%
% The following two commands are all you need in the
% initial runs of your .tex file to
% produce the bibliography for the citations in your paper.
\bibliographystyle{abbrv}
\bibliography{sources}  % sigproc.bib is the name of the Bibliography in this case
% You must have a proper ".bib" file
%  and remember to run:
% latex bibtex latex latex
% to resolve all references
%
% ACM needs 'a single self-contained file'!
%
%APPENDICES are optional
%\balancecolumns
\balancecolumns
% That's all folks!
\end{document}
